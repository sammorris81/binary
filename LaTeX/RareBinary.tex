\documentclass[11pt]{article}
\usepackage{amssymb, amsthm, amsmath}
\usepackage{bm}
\usepackage{graphicx}
\usepackage[authoryear]{natbib}
\usepackage{bm}
\usepackage{verbatim}
\usepackage{lineno}
\usepackage{times}
\usepackage{soul}
\usepackage{color}
\usepackage{enumitem}
\usepackage{setspace}
\usepackage{times}

\usepackage[left=1in,top=1in,right=1in]{geometry}
\pdfpageheight 11in
\pdfpagewidth 8.5in
\linespread{2.0}
\newcommand{\btheta}{ \mbox{\boldmath $\theta$}}
\newcommand{\bmu}{ \mbox{\boldmath $\mu$}}
\newcommand{\balpha}{ \mbox{\boldmath $\alpha$}}
\newcommand{\bbeta}{ \mbox{\boldmath $\beta$}}
\newcommand{\bdelta}{ \mbox{\boldmath $\delta$}}
\newcommand{\blambda}{ \mbox{\boldmath $\lambda$}}
\newcommand{\bgamma}{ \mbox{\boldmath $\gamma$}}
\newcommand{\brho}{ \mbox{\boldmath $\rho$}}
\newcommand{\bpsi}{ \mbox{\boldmath $\psi$}}
\newcommand{\bepsilon}{ \mbox{\boldmath $\epsilon$}}
\newcommand{\bomega}{ \mbox{\boldmath $\omega$}}
\newcommand{\bOmega}{ \mbox{\boldmath $\Omega$}}
\newcommand{\bDelta}{ \mbox{\boldmath $\Delta$}}
\newcommand{\bSigma}{ \mbox{\boldmath $\Sigma$}}
\newcommand{\bPsi}{\mbox{\boldmath $\Psi$}}
\newcommand{\bOne}{\mbox{\boldmath $1$}}
\newcommand{\omu}{\overline{\mu}}
\newcommand{\oSigma}{\overline{\Sigma}}
\newcommand{\Yt}{{\tilde Y}}
\newcommand{\bA}{ \mbox{\bf A}}
\newcommand{\bP}{ \mbox{\bf P}}
\newcommand{\bx}{ \mbox{\bf x}}
\newcommand{\bX}{ \mbox{\bf X}}
\newcommand{\bB}{ \mbox{\bf B}}
\newcommand{\bZ}{ \mbox{\bf Z}}
\newcommand{\by}{ \mbox{\bf y}}
\newcommand{\bY}{ \mbox{\bf Y}}
\newcommand{\bz}{ \mbox{\bf z}}
\newcommand{\bh}{ \mbox{\bf h}}
\renewcommand{\bm}{ \mbox{\bf m}}
\newcommand{\br}{ \mbox{\bf r}}
\newcommand{\bt}{ \mbox{\bf t}}
\newcommand{\bs}{ \mbox{\bf s}}
\newcommand{\bb}{ \mbox{\bf b}}
\newcommand{\bL}{ \mbox{\bf L}}
\newcommand{\bu}{ \mbox{\bf u}}
\newcommand{\bv}{ \mbox{\bf v}}
\newcommand{\bV}{ \mbox{\bf V}}
\newcommand{\bW}{ \mbox{\bf W}}
\newcommand{\bG}{ \mbox{\bf G}}
\newcommand{\bH}{ \mbox{\bf H}}
\newcommand{\bw}{ \mbox{\bf w}}
\newcommand{\bo}{ \mbox{\bf o}}
\newcommand{\bfe}{ \mbox{\bf e}}
\newcommand{\iid}{\stackrel{iid}{\sim}}
\newcommand{\ind}{\stackrel{ind}{\sim}}
\newcommand{\dd}{\; \text{d} }
\newcommand{\ddd}{\text{d} }
\newcommand{\indep}{\stackrel{indep}{\sim}}
\newcommand{\converged}{\stackrel{d}{\rightarrow}}
\newcommand{\calR}{{\cal R}}
\newcommand{\calG}{{\cal G}}
\newcommand{\calD}{{\cal D}}
\newcommand{\calS}{{\cal S}}
\newcommand{\calB}{{\cal B}}
\newcommand{\calA}{{\cal A}}
\newcommand{\calT}{{\cal T}}
\newcommand{\calO}{{\cal O}}
\newcommand{\argmax}{{\mathop{\rm arg\, max}}}
\newcommand{\argmin}{{\mathop{\rm arg\, min}}}
\newcommand{\Frechet}{\mbox{Fr$\acute{\mbox{e}}$chet }}
\newcommand{\Matern}{\mbox{Mat$\acute{\mbox{e}}$rn }}
\newcommand{\ballunion}{B_a(\bs_1) \cup B_b(\bs_2) }

\newcommand{\beq}{ \begin{equation}}
\newcommand{\eeq}{ \end{equation}}
\newcommand{\beqn}{ \begin{eqnarray}}
\newcommand{\eeqn}{ \end{eqnarray}}

\newcommand*\patchAmsMathEnvironmentForLineno[1]{%
  \expandafter\let\csname old#1\expandafter\endcsname\csname #1\endcsname
  \expandafter\let\csname oldend#1\expandafter\endcsname\csname end#1\endcsname
  \renewenvironment{#1}%
     {\linenomath\csname old#1\endcsname}%
     {\csname oldend#1\endcsname\endlinenomath}}%
\newcommand*\patchBothAmsMathEnvironmentsForLineno[1]{%
  \patchAmsMathEnvironmentForLineno{#1}%
  \patchAmsMathEnvironmentForLineno{#1*}}%
\AtBeginDocument{%
\patchBothAmsMathEnvironmentsForLineno{equation}%
\patchBothAmsMathEnvironmentsForLineno{align}%
\patchBothAmsMathEnvironmentsForLineno{flalign}%
\patchBothAmsMathEnvironmentsForLineno{alignat}%
\patchBothAmsMathEnvironmentsForLineno{gather}%
\patchBothAmsMathEnvironmentsForLineno{multline}%
}



\begin{document}\linenumbers

\begin{center}
{\Large {\bf A spatial model for rare binary events}}\\
\today
\end{center}

\section{Introduction}\label{s:intro}

\section{Binary regression}\label{s:model}
Let $Y_i\in\{0,1\}$ be the binary response at spatial location $\bs_i \in \calD$, and $\bX_i$ be the associated $p$-vector of covariates with first element equal to one for the intercept.
The goal in binary regression is to relate a set of covariates with the response using the link function $g$ so that P$(Y_i=1) = \pi_i= g(\bX_i \bbeta)$, where $\bX_i$ is the vector of covariates for observation $i$, and $\bbeta$ is the $p$-vector of regression coefficients.
Two very commonly used types of binary regression include logistic regression with
\begin{align}
  \pi_i = \frac{ \exp{\bX_i \bbeta} }{1 + \exp{\bX_i \bbeta}}
\end{align}
and probit regression with
\begin{align}
  \pi_i = \Phi(\bX_i \bbeta)
\end{align}
where $\Phi(\cdot)$ represents the standard normal distribution function.
One challenge to these link functions is that they are symmetric, so they may not be appropriate in the case of asymmetric data.
More recently, \cite{Wang-2010} introduced the generalized extreme value (GEV) link function for rare binary data where
\begin{align}
  \pi_i= 1-\exp\left[\left(1-\xi\bX_i\bbeta\right)^{-1/\xi}\right].
\end{align}
In the case that $\xi = 0$, this is the complementary log-log (cloglog) link function.
The GEV link function is attractive because it is not asymmetric, and therefore can give a more flexible fit to the data.

\subsection{Spatial dependence}\label{s:spatialdependence}
For logistic regression

We propose a copula \citep{nelsen-1999} to account for spatial dependence while preserving the marginal event probabilities. Let $Y_i = I(Z_i>z_i)$, where $Z_i$ is a continuous latent variable and $z_i$ is the appropriate threshold so that $P(Y_i=1)=p_i$.  The latent $Z_i$ is modeled using spatial extreme value analysis methods to capture dependence between rare events.  We assume $Z$ follows the max-stable spatial process of \cite{reich-2012}.  Under this model, the marginal distribution of each $Z_i$ is GEV(1,1,1) with P$(Z_i>c) = 1-\exp(-1/c)$.  Therefore, we must set $z_i=-1/\log(1-p_i)$ so that $P(Y_i=1)=p_i$.

Spatial dependence is determined by the joint distribution of $\bZ = (Z_1,\ldots,Z_n)$,
\beq\label{jointCDF}
 G(\bz) =  \mbox{P}[Z_1<z_1,\ldots,Z_n<z_n] = \exp\left\{-\sum_{l=1}^L\left[\sum_{i=1}^n\left(\frac{w_{l}(\bs_i)}{z_i}\right)^{1/\alpha}\right]^{\alpha}\right\},
\eeq
where $\bz = (z_1,\ldots,z_n)$. This is a special case of the multivariate GEV distribution with asymmetric Laplace dependence function \citep{Tawn-1990}.  The parameter $\alpha\in(0,1)$ determines the strength of dependence, with $\alpha$ near zero giving strong dependence and $\alpha=1$ giving joint independence. The weights $w_{li}>0$ determine the spatial dependence structure, and are discussed in detail in Section \ref{s:spatial}.  Many weight functions are possible, but the weights must be constrained so that $\sum_{l=1}^L w_{l}(\bs_i)=1$ for all $i=1,\ldots,n$ to preserve the marginal GEV distribution.

\section{Spatial dependence}\label{s:spatial}

The weights $w_{l}(\bs_i)$ in (\ref{jointCDF}) should vary smoothly across space to induce spatial dependence.  For example, \cite{reich-2012} take the weights to be scaled Gaussian kernels with knots $\bv_l$, that is
\beq\label{w}
   w_{l}(\bs_i) = \frac{\exp\left[-0.5\left(||\bs_i-\bv_l||/\rho\right)^2\right]}
                 {\sum_{j=1}^L\exp\left[-0.5\left(||\bs_i-\bv_j||/\rho\right)^2\right]}.
\eeq
To kernel bandwidth $\rho>0$ determines the spatial range of the dependence, with large $\rho$ giving long-range dependence and vice versa.

Then in a bivariate setting, the probability of observing a joint exceedances as a function of $\alpha$ is

\begin{align} \label{biv}
  \text{P}(Y_i = 1, & Y_j = 1) = 1 - \exp\left\{ - \frac{ 1 }{ z_i } \right\} - \exp \left\{ - \frac{ 1 }{z_j} \right\} + \exp \left\{ - \sum_{ l = 1 }^{ L } \left[ \left( \frac{ w_{ l }(\bs_i) }{ z_i } \right)^{1/\alpha} + \left( \frac{ w_{l }(\bs_i)}{z_j} \right)^{1/\alpha} \right]^{\alpha} \right\} \nonumber \\
  		&= p_i + p_j - \left( 1 - \exp \left \{ - \sum_{ l = 1 }^{ L } \left( \left[ -\log(1-p_i) w_{l}(\bs_i) \right]^{ 1/\alpha } + \left[ -\log(1 - p_j) w_{ l}(\bs_j) \right]^{ 1/\alpha } \right)^{\alpha} \right \} \right).
\end{align}

To describe the tail dependence, we use the $\chi$ statistic of \citet{Coles-1999}.
Assume that $Y_i$ and $Y_j$ have the same marginal distributions, then $p_i = p_j = p$ for all $i, j$.
As shown in Appendix A.2,

\begin{align}
\chi = 2 -  \vartheta(\bs_i, \bs_j).
\end{align}
where $\vartheta (\bs_i, \bs_j) = \sum_{ l = 1 }^{ L } \left[ w_{l}(\bs_i)^{ 1/\alpha } +  w_{ l}(\bs_j)^{ 1/\alpha } \right]^\alpha$ is the pairwise extremal coefficient given by \citet{reich-2012}.
In the case of complete dependence, $\chi = 1$, and in the case of complete independence, $\chi = 0$.
This is relatively easy to show for $\alpha = 1$, but I don't know of a way to prove $\lim_{\alpha \rightarrow 0} \chi = 1$. Any thoughts?

\section{Computation}\label{s:comp}

As shown in Appendix A.1, the joint probability mass function of $\bY=(Y_1,\ldots,Y_n)$ has a convenient form when the number of events is small.  Let $K=\sum_{i=1}^nY_i$ be the number of events, and assume without loss of generality the data are ordered so that the $Y_1=\ldots=Y_K=1$.  Then
\beq\label{pmf}
   P(Y_1=y_1,\ldots,Y_n=y_n) =  \left\{
                               \begin{array}{ll}
                                 G(\bz) & K=0 \\
                                 G(\bz_{(1)})-G(\bz) & K=1 \\
                                 G(\bz_{(12)})-G(\bz_{(1)})-G(\bz_{(2)})+G(\bz) & K=2
                               \end{array}
                             \right.
\eeq
where $G(\bz_{(1)}) = P(Z_2<z_2,\ldots,Z_n<z_n)$, $G(\bz_{(2)}) = P(Z_1<z_1,Z_3<z_3,\ldots,Z_n<z_n)$, and $G(\bz_{(12)}) = P(Z_3<z_3,\ldots,Z_n<z_n)$.  Similar expressions can be derived for all $K$, but become cumbersome for large $K$.  Therefore, for small $K$ we can evaluate the likelihood directly.  Most days in our dataset have $K<4$, so we use this expression for those days.  However for days with many events, we must use the latent variable scheme described below (unless you can think of a better way!).
I think it should be more computationally efficient to use (\ref{pmf}) for any $K$. At most, we have to calculate the $\left( \frac{ w_{l}(\bs_i) }{ z_i } \right)^{1/\alpha}$, for all $i, l$. In the random effects model, the expression for the joint density conditional on $\theta$ is
\beq
	P(Y_1=y_1,\ldots,Y_n=y_n) = \prod_{ i = 1 }^{ n } \left[ \exp \left\{ \sum_{ l = 1 }^{L} A_l \left( \frac{ w_{l}(\bs_i) }{ z_i } \right)^{ 1/\alpha} \right\} \right]^{ 1 - Y_i } \left[ 1 - \exp \left\{ \sum_{ l = 1 }^{L} A_l \left( \frac{ w_{l}(\bs_i) }{ z_i } \right)^{ 1/\alpha} \right\} \right]^{ Y_i }.
\eeq
So we still need to compute $\left( \frac{ w_{l}(\bs_i) }{ z_i } \right)^{1/\alpha}$, but we also need to do the sampling for all the $A_l$ terms as well.

\section{Simulation study}\label{s:sim}

\section{Data analysis}\label{s:analysis}

\section{Conclusions}\label{s:con}

\section*{Acknowledgments}

\section*{Appendix A.1: Derivation of the likelihood}
We use the hierarchical max-stable spatial model given by \citet{reich-2012}. If at each margin, $Z_i \sim $ GEV$(1,1,1)$, then $Z_i | \theta_i \indep $ GEV$(\theta, \alpha \theta, \alpha)$. As defined in section \ref{s:comp}, we reorder the data such that $Y_1=\ldots=Y_K=1$, and $Y_{K+1} = \ldots = Y_n = 0$. Then the joint likelihood conditional on the random effect $\theta$ is

\begin{align} \label{joint_cond}
	P(Y_1=y_1,\ldots,Y_n=y_n) &= \prod_{ i \le K } \left\{ 1 - \exp \left[ - \left( \frac{ \theta_i }{ z_i } \right)^{ 1/\alpha} \right] \right \} \prod_{ i > K } \exp \left[ -\left( \frac{ \theta_i }{ z_i } \right)^{1/\alpha} \right] \nonumber \\[0.5em]
		&= \exp \left[ -\sum_{ i = K+1}^{ n }\left( \frac{ \theta_i }{ z_i } \right)^{1/\alpha} \right] - \exp \left[ -\sum_{ i = K+1}^{ n }\left( \frac{ \theta_i }{ z_i } \right)^{1/\alpha} \right] \sum_{ i = 1}^{K} \exp\left[ -\left( \frac{ \theta_i }{ z_i } \right)^{ 1/\alpha} \right] \nonumber\\
		&  + \exp \left[ -\sum_{ i = K+1}^{ n }\left( \frac{ \theta_i }{ z_i } \right)^{1/\alpha} \right] \sum_{ 1 < i < j \le K } \left\{ \exp \left[ - \left( \frac{ \theta_i }{ z_i } \right)^{ 1/\alpha} - \left( \frac{ \theta_j }{ z_j } \right)^{ 1/\alpha } \right] \right \} \nonumber \\[0.5em]
		& + \cdots + (-1)^K \exp\left[ - \sum_{ i = 1 }^{ n }\left( \frac{ \theta_i }{ z_i } \right)^{ 1/\alpha} \right]
\end{align}

Finally marginalizing over the random effect, we obtain

\begin{align} \label{joint}
    P(Y_1=y_1,\ldots,Y_n=y_n) &=\int G(\bz | \bA) p( \bA | \alpha) d\bA. \nonumber\\[0.5em]
			&= \int \exp \left[ -\sum_{ i = K+1}^{ n }\left( \frac{ \theta_i }{ z_i } \right)^{1/\alpha} \right] - \exp \left[ -\sum_{ i = K+1}^{ n }\left( \frac{ \theta_i }{ z_i } \right)^{1/\alpha} \right] \sum_{ i = 1}^{K} \exp\left[ -\left( \frac{ \theta_i }{ z_i } \right)^{ 1/\alpha} \right] \nonumber\\
		&  + \exp \left[ -\sum_{ i = K+1}^{ n }\left( \frac{ \theta_i }{ z_i } \right)^{1/\alpha} \right] \sum_{ 1 < i < j \le K } \left\{ \exp \left[ - \left( \frac{ \theta_i }{ z_i } \right)^{ 1/\alpha} - \left( \frac{ \theta_j }{ z_j } \right)^{ 1/\alpha } \right] \right \} \nonumber \\[0.5em]
		& + \cdots + (-1)^K \exp\left[ - \sum_{ i = 1 }^{ n }\left( \frac{ \theta_i }{ z_i } \right)^{ 1/\alpha} \right] p( \bA | \alpha) d\bA.
\end{align}

Consider the first term in the summation,

\begin{align}
	\int \exp \left\{ -\sum_{ i = K+1}^{ n }\left( \frac{ \theta_i }{ z_i } \right)^{1/\alpha} \right\} p( \bA | \alpha) d\bA &= \int \exp \left\{ - \sum_{ i = K + 1 }^n \left( \frac{ \left[ \sum_{ l = 1 }^L  A_l w_{l}(\bs_i)^{1/\alpha} \right)^\alpha }{ z_i} \right]^{ 1/\alpha } \right \} p( \bA | \alpha) d\bA \nonumber \\[0.5em]
	 &= \int \exp \left\{ -\sum_{ i = K + 1}^n \sum_{ l = 1}^L A_l \left( \frac{ w_l (\bs_i) }{ z_i } \right)^{1/\alpha} \right \} p( \bA | \alpha) d\bA \nonumber \\[0.5em]
	 &=\exp\left\{-\sum_{ l = 1}^L \left[ \sum_{ i = K + 1 }^n \left( \frac{ w_l(\bs_i)}{ z_i} \right)^{1/\alpha} \right]^\alpha \right\}.
\end{align}

The remaining terms in equation (\ref{joint}) are straightforward to obtain, and after integrating out the random effect, the joint density is the density given in (\ref{pmf}).

\section*{Appendix A.2: Derivation of the $\chi$ statistic}
\begin{align} \label{chi}
  \chi &= \lim_{p \rightarrow 0 }\text{P}(Y_i = 1|Y_j = 1)\nonumber \\
   &= \lim_{p \rightarrow \infty }\frac{ p + p - \left( 1 - \exp \left \{ - \sum_{ l = 1 }^{ L } \left[ \left( -\log(1-p) w_{l}(\bs_i) \right)^{ 1/\alpha } + \left( -\log(1 - p) w_{ l}(\bs_j) \right)^{ 1/\alpha } \right]^{\alpha} \right \} \right) }{ p } \nonumber \\[0.5em]
  &= \lim_{p \rightarrow 0 } \frac{ 2p - \left( 1 - \exp \left \{ \log(1-p) \sum_{ l = 1 }^{ L } \left[  w_{l}(\bs_i) ^{ 1/\alpha } +  w_{ l}(\bs_j) ^{ 1/\alpha } \right]^{\alpha} \right \} \right) }{ p } \nonumber \\[0.5em]
  &= \lim_{p \rightarrow 0 } \frac{ 2p - \left( 1 - (1-p)^{ \sum_{ l = 1 }^{ L } \left[ \left( w_{l}(\bs_i) \right)^{ 1/\alpha } + \left( w_{ l}(\bs_j) \right)^{ 1/\alpha } \right]^\alpha } \right) }{ p } \nonumber \\[0.5em]
  &=\lim_{p \rightarrow 0 } 2 -\sum_{ l = 1 }^{ L } \left[ w_{l}(\bs_i) ^{ 1/\alpha } +  w_{ l} (\bs_j)^{ 1/\alpha } \right]^\alpha ( 1 - p)^{ -1 + \sum_{ l = 1 }^{ L } \left[  w_{l}(\bs_i) ^{ 1/\alpha } +  w_{ l}(\bs_j)^{ 1/\alpha } \right]^\alpha } \nonumber \\[0.5em]
  &= 2 -  \sum_{ l = 1 }^{ L } \left[ w_{l}(\bs_i)^{ 1/\alpha } +  w_{ l}(\bs_j) ^{ 1/\alpha } \right]^\alpha.
\end{align}

\begin{singlespace}
\bibliographystyle{rss}
\bibliography{spatialbinary}
\end{singlespace}


\end{document}

