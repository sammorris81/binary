\documentclass[11pt]{article}
\usepackage{amssymb, amsthm, amsmath}
\usepackage{bm}
\usepackage{graphicx}
\usepackage[authoryear]{natbib}
\usepackage{bm}
\usepackage{verbatim}
\usepackage{lineno}
\usepackage{times}
\usepackage{soul}
\usepackage{color}
\usepackage{enumitem}
\usepackage{setspace}
\usepackage{times}

\usepackage[left=1in,top=1in,right=1in]{geometry}
\pdfpageheight 11in
\pdfpagewidth 8.5in
\linespread{2.0}
\newcommand{\btheta}{ \mbox{\boldmath $\theta$}}
\newcommand{\bmu}{ \mbox{\boldmath $\mu$}}
\newcommand{\balpha}{ \mbox{\boldmath $\alpha$}}
\newcommand{\bbeta}{ \mbox{\boldmath $\beta$}}
\newcommand{\bdelta}{ \mbox{\boldmath $\delta$}}
\newcommand{\blambda}{ \mbox{\boldmath $\lambda$}}
\newcommand{\bgamma}{ \mbox{\boldmath $\gamma$}}
\newcommand{\brho}{ \mbox{\boldmath $\rho$}}
\newcommand{\bpsi}{ \mbox{\boldmath $\psi$}}
\newcommand{\bepsilon}{ \mbox{\boldmath $\epsilon$}}
\newcommand{\bomega}{ \mbox{\boldmath $\omega$}}
\newcommand{\bOmega}{ \mbox{\boldmath $\Omega$}}
\newcommand{\bDelta}{ \mbox{\boldmath $\Delta$}}
\newcommand{\bSigma}{ \mbox{\boldmath $\Sigma$}}
\newcommand{\bPsi}{\mbox{\boldmath $\Psi$}}
\newcommand{\bOne}{\mbox{\boldmath $1$}}
\newcommand{\omu}{\overline{\mu}}
\newcommand{\oSigma}{\overline{\Sigma}}
\newcommand{\Yt}{{\tilde Y}}
\newcommand{\bA}{ \mbox{\bf A}}
\newcommand{\bP}{ \mbox{\bf P}}
\newcommand{\bx}{ \mbox{\bf x}}
\newcommand{\bX}{ \mbox{\bf X}}
\newcommand{\bB}{ \mbox{\bf B}}
\newcommand{\bZ}{ \mbox{\bf Z}}
\newcommand{\by}{ \mbox{\bf y}}
\newcommand{\bY}{ \mbox{\bf Y}}
\newcommand{\bz}{ \mbox{\bf z}}
\newcommand{\bh}{ \mbox{\bf h}}
\renewcommand{\bm}{ \mbox{\bf m}}
\newcommand{\br}{ \mbox{\bf r}}
\newcommand{\bt}{ \mbox{\bf t}}
\newcommand{\bs}{ \mbox{\bf s}}
\newcommand{\bb}{ \mbox{\bf b}}
\newcommand{\bL}{ \mbox{\bf L}}
\newcommand{\bu}{ \mbox{\bf u}}
\newcommand{\bv}{ \mbox{\bf v}}
\newcommand{\bV}{ \mbox{\bf V}}
\newcommand{\bW}{ \mbox{\bf W}}
\newcommand{\bG}{ \mbox{\bf G}}
\newcommand{\bH}{ \mbox{\bf H}}
\newcommand{\bw}{ \mbox{\bf w}}
\newcommand{\bo}{ \mbox{\bf o}}
\newcommand{\bfe}{ \mbox{\bf e}}
\newcommand{\iid}{\stackrel{iid}{\sim}}
\newcommand{\ind}{\stackrel{ind}{\sim}}
\newcommand{\dd}{\; \text{d} }
\newcommand{\ddd}{\text{d} }
\newcommand{\indep}{\stackrel{indep}{\sim}}
\newcommand{\converged}{\stackrel{d}{\rightarrow}}
\newcommand{\calR}{{\cal R}}
\newcommand{\calG}{{\cal G}}
\newcommand{\calD}{{\cal D}}
\newcommand{\calS}{{\cal S}}
\newcommand{\calB}{{\cal B}}
\newcommand{\calA}{{\cal A}}
\newcommand{\calT}{{\cal T}}
\newcommand{\calO}{{\cal O}}
\newcommand{\argmax}{{\mathop{\rm arg\, max}}}
\newcommand{\argmin}{{\mathop{\rm arg\, min}}}
\newcommand{\Frechet}{\mbox{Fr$\acute{\mbox{e}}$chet }}
\newcommand{\Matern}{\mbox{Mat$\acute{\mbox{e}}$rn }}
\newcommand{\ballunion}{B_a(\bs_1) \cup B_b(\bs_2) }

\newcommand{\beq}{ \begin{equation}}
\newcommand{\eeq}{ \end{equation}}
\newcommand{\beqn}{ \begin{eqnarray}}
\newcommand{\eeqn}{ \end{eqnarray}}

\newcommand*\patchAmsMathEnvironmentForLineno[1]{%
  \expandafter\let\csname old#1\expandafter\endcsname\csname #1\endcsname
  \expandafter\let\csname oldend#1\expandafter\endcsname\csname end#1\endcsname
  \renewenvironment{#1}%
     {\linenomath\csname old#1\endcsname}%
     {\csname oldend#1\endcsname\endlinenomath}}%
\newcommand*\patchBothAmsMathEnvironmentsForLineno[1]{%
  \patchAmsMathEnvironmentForLineno{#1}%
  \patchAmsMathEnvironmentForLineno{#1*}}%
\AtBeginDocument{%
\patchBothAmsMathEnvironmentsForLineno{equation}%
\patchBothAmsMathEnvironmentsForLineno{align}%
\patchBothAmsMathEnvironmentsForLineno{flalign}%
\patchBothAmsMathEnvironmentsForLineno{alignat}%
\patchBothAmsMathEnvironmentsForLineno{gather}%
\patchBothAmsMathEnvironmentsForLineno{multline}%
}



\begin{document}\linenumbers

\begin{center}
{\Large {\bf A spatial model for rare binary events}}\\
\today
\end{center}

\section{Introduction}\label{s:intro}

\section{Binary regression}\label{s:model}
Let $Y_i\in\{0,1\}$ be the binary response at spatial location $\bs_i \in \calD$, and $\bX_i$ be the associated $p$-vector of covariates with first element equal to one for the intercept.
The goal in binary regression is to relate a set of covariates with the response using the link function $g$ so that P$(Y_i=1) = \pi_i= g(\bX_i \bbeta)$, where $\bX_i$ is the vector of covariates for observation $i$, and $\bbeta$ is the $p$-vector of regression coefficients.
Two very commonly used types of binary regression include logistic regression with $\pi_i = \frac{ \exp{\bX_i \bbeta} }{1 + \exp{\bX_i \bbeta}}$ and probit regression with $\pi_i = \Phi(\bX_i \bbeta)$ where $\Phi(\cdot)$ represents the standard normal distribution function.
One limitation to these link functions is that they assume the data are symmetric which may not always be the case.
More recently, \citet{Wang2010} introduced the generalized extreme value (GEV) link function for rare binary data where
\begin{align} \label{eq:marginalz}
  \pi_i= 1 - \exp \left[-\left(1 - \xi \bX_i \bbeta \right)^{-1 / \xi} \right].
\end{align}
In the case that $\xi = 0$, this is the complementary log-log (cloglog) link function.
The GEV link function is attractive because it allows for assymetry thereby providing a more flexible model for the data.

\section{Spatial dependence}\label{s:spatialdependence}
In this section we extend the GEV link function to allow for spatial correlation.
First, to simplify the notation, we let $z_i = \left(1 - \xi \bX_i \bbeta \right)^{1 / \xi}$ from (\ref{eq:marginalz}).
To incorpoate spatial dependence into the model, we consider the hierarchical max-stable process of \citet{Reich2012}.
The spatial dependence is determined by the joint distribution of $\bZ = (Z_1,\ldots,Z_n)$,
\begin{align}\label{jointCDF}
 G(\bz) =  \mbox{P}[Z_1<z_1,\ldots,Z_n<z_n] = \exp\left\{-\sum_{l=1}^L\left[\sum_{i=1}^n\left(\frac{w_{l}(\bs_i)}{z_i}\right)^{1/\alpha}\right]^{\alpha}\right\},
\end{align}
where $\bz = (z_1,\ldots,z_n)$, $w_{l}(\bs_i)$ are a set of weights that determine the spatial dependence structure and are discussed further in Section \ref{s:weights}, and $\alpha\in(0,1)$ determines the strength of dependence, with $\alpha$ near zero giving strong dependence and $\alpha=1$ giving joint independence.
This is a special case of the multivariate GEV distribution with asymmetric Laplace dependence function \citep{Tawn1990}.
One nice feature to this hierarchical model is that the lower-dimensional marginal distributions also follow a multivariate extreme value distribution.
More importantly, at a single site $i$, the marginal distribution gives \mbox{$P(Y_i = 1) = 1 - \exp\left\{ -\frac{ 1 }{ z_i} \right\}$} which is the same as the marginal distributions given by \citet{Wang2010}.

\subsection{Weight functions}\label{s:weights}
Many weight functions are possible, but the weights must be constrained so that $\sum_{l=1}^L w_{l}(\bs_i)=1$ for all $i=1,\ldots,n$ to preserve the marginal GEV distribution.
The weights $w_{l}(\bs_i)$ in (\ref{jointCDF}) should vary smoothly across space to induce spatial dependence.
For example, \cite{Reich2012} take the weights to be scaled Gaussian kernels with knots $\bv_l$, that is
\begin{align}\label{w}
   w_{l}(\bs_i) = \frac{\exp\left[-0.5\left(||\bs_i-\bv_l||/\rho\right)^2\right]}
                 {\sum_{j=1}^L\exp\left[-0.5\left(||\bs_i-\bv_j||/\rho\right)^2\right]}.
\end{align}
The kernel bandwidth $\rho>0$ determines the spatial range of the dependence, with large $\rho$ giving long-range dependence and vice versa.

\section{Multivariate distribution}\label{s:multivariate}
As shown in Appendix \ref{a:likelihoodderivation}, the joint probability mass function of $\bY=(Y_1,\ldots,Y_n)$ has a convenient form when the number of events is small.
Let $K=\sum_{i=1}^nY_i$ be the number of events, and assume without loss of generality the data are ordered so that the $Y_1=\ldots=Y_K=1$.
Then
\begin{align}\label{pmf}
  P(Y_1=y_1,\ldots,Y_n=y_n) =  \left\{
    \begin{array}{ll}
      G(\bz) & K=0 \\
      G(\bz_{(1)})-G(\bz) & K=1 \\
      G(\bz_{(12)})-G(\bz_{(1)})-G(\bz_{(2)})+G(\bz) & K=2
    \end{array}
  \right.
\end{align}
where $G(\bz_{(1)}) = P(Z_2<z_2,\ldots,Z_n<z_n)$, $G(\bz_{(2)}) = P(Z_1<z_1,Z_3<z_3,\ldots,Z_n<z_n)$, and $G(\bz_{(12)}) = P(Z_3<z_3,\ldots,Z_n<z_n)$.
Similar expressions can be derived for all $K$, but become cumbersome for large $K$.

\subsection{Bivariate distribution}\label{s:bivariate}
Then in a bivariate setting, the probability of observing a joint exceedances as a function of $\alpha$ is
\begin{align} \label{biv}
  \text{P}(Y_i = 1, & Y_j = 1) = 1 - \exp\left\{ - \frac{ 1 }{ z_i } \right\} - \exp \left\{ - \frac{ 1 }{z_j} \right\} + \exp \left\{ - \sum_{ l = 1 }^{ L } \left[ \left( \frac{ w_{ l }(\bs_i) }{ z_i } \right)^{1/\alpha} + \left( \frac{ w_{l }(\bs_i)}{z_j} \right)^{1/\alpha} \right]^{\alpha} \right\}
\end{align}
In the literature on extremes, one common metric to describe the bivariate dependence is the $\chi$ statistic of \citet{Coles1999}.
The $\chi$ statistic between two observations $z_1$ and $z_2$ is given by
\begin{align}
  \chi(\bs_1, \bs_2) = \lim_{c \rightarrow \infty} = P(Z_1 > c | Z_2 > c).
\end{align}
However, in this latent variable approach, $\lim_{c \rightarrow \infty}$ may not be the most reasonable metric because the observed data are a series of zeros and ones.
Therefore, we chose the $\kappa$ statistic of \citet{Cohen1960} defined by
\begin{align}
  \kappa = \frac{P(A) - P(E)}{1 - P(E)}
\end{align}
where $P(A)$ is the joint probability of agreement and $P(E)$ is the joint probability of agreement under an assumption of independence.
We believe this measure of dependence to be reasonable because,
\begin{align}
  \lim_{\beta_0 \rightarrow \infty} \kappa(h) = \chi(h) = 2 - \vartheta(\bs_i, \bs_j)
\end{align}
where $\beta_0$ is the intercept from $\bX^T\bbeta$ and $\vartheta (\bs_i, \bs_j) = \sum_{ l = 1 }^{ L } \left[ w_{l}(\bs_i)^{ 1/\alpha } +  w_{ l}(\bs_j)^{ 1/\alpha } \right]^\alpha$ is the pairwise extremal coefficient given by \citet{Reich2012} (see Appendix \ref{a:chi}).
In the case of complete dependence, $\kappa = 1$, and in the case of complete independence, $\kappa = 0$.

\section{Computation}\label{s:comp}
For small $K$ we can evaluate the likelihood directly.
In the random effects model, the expression for the joint density conditional on $\theta$ is
\begin{align}
	P(Y_1=y_1,\ldots,Y_n=y_n) = \prod_{ i = 1 }^{ n } \left[ \exp \left\{ \sum_{ l = 1 }^{L} A_l \left( \frac{ w_{l}(\bs_i) }{ z_i } \right)^{ 1/\alpha} \right\} \right]^{ 1 - Y_i } \left[ 1 - \exp \left\{ \sum_{ l = 1 }^{L} A_l \left( \frac{ w_{l}(\bs_i) }{ z_i } \right)^{ 1/\alpha} \right\} \right]^{ Y_i }.
\end{align}

\section{Simulation study}\label{s:sim}
For our simulation study, we generate $n_m = 50$ datasets under 8 different settings to explore the impact of spatial dependence, rareness of observations, and misspecification of link function.
We consider cases of high and low dependence; two degrees of rareness $\pi = 0.01, 0.05$; and two underlying link functions, logit and GEV.
For each dataset, we fit the model using three different methods, spatial logistic regression, spatial probit regression, and the proposed spatial GEV method.
In each case, we fit the model using Bayesian methods with proper, but fairly uninformative priors.
In particular, when selecting a model, we consider how well the method does at estimating the  $\kappa$ function

\section{Data analysis}\label{s:analysis}
For the data analysis, we consider data from the eBirds dataset, a citizen-based observation network of bird sitings in the United States \citep{Sullivan2009}.
The data are publicly available from {\tt http://ebird.org}.
We use data from 2002, and focus specifically on cattle egrets and sanderlings.

\section{Conclusions}\label{s:con}

\section*{Acknowledgments}

\appendix
\section{Appendices}

\subsection{Derivation of the likelihood} \label{a:likelihoodderivation}
We use the hierarchical max-stable spatial model given by \citet{Reich2012}. If at each margin, $Z_i \sim $ GEV$(1,1,1)$, then $Z_i | \theta_i \indep $ GEV$(\theta, \alpha \theta, \alpha)$. As defined in section \ref{s:comp}, we reorder the data such that $Y_1=\ldots=Y_K=1$, and $Y_{K+1} = \ldots = Y_n = 0$. Then the joint likelihood conditional on the random effect $\theta$ is

\begin{align} \label{joint_cond}
	P(Y_1=y_1,\ldots,Y_n=y_n) =& \prod_{ i \le K } \left\{ 1 - \exp \left[ - \left( \frac{ \theta_i }{ z_i } \right)^{ 1/\alpha} \right] \right \} \prod_{ i > K } \exp \left[ -\left( \frac{ \theta_i }{ z_i } \right)^{1/\alpha} \right] \nonumber \\[0.5em]
		=& \exp \left[ -\sum_{ i = K+1}^{ n }\left( \frac{ \theta_i }{ z_i } \right)^{1/\alpha} \right] - \exp \left[ -\sum_{ i = K+1}^{ n }\left( \frac{ \theta_i }{ z_i } \right)^{1/\alpha} \right] \sum_{ i = 1}^{K} \exp\left[ -\left( \frac{ \theta_i }{ z_i } \right)^{ 1/\alpha} \right] \nonumber\\
		&  + \exp \left[ -\sum_{ i = K+1}^{ n }\left( \frac{ \theta_i }{ z_i } \right)^{1/\alpha} \right] \sum_{ 1 < i < j \le K } \left\{ \exp \left[ - \left( \frac{ \theta_i }{ z_i } \right)^{ 1/\alpha} - \left( \frac{ \theta_j }{ z_j } \right)^{ 1/\alpha } \right] \right \} \nonumber \\[0.5em]
		& + \cdots + (-1)^K \exp\left[ - \sum_{ i = 1 }^{ n }\left( \frac{ \theta_i }{ z_i } \right)^{ 1/\alpha} \right]
\end{align}

Finally marginalizing over the random effect, we obtain

\begin{align} \label{joint}
    P(Y_1=y_1,\ldots,Y_n=y_n) =&\int G(\bz | \bA) p( \bA | \alpha) d\bA. \nonumber\\[0.5em]
			=& \int \exp \left[ -\sum_{ i = K+1}^{ n }\left( \frac{ \theta_i }{ z_i } \right)^{1/\alpha} \right] - \exp \left[ -\sum_{ i = K+1}^{ n }\left( \frac{ \theta_i }{ z_i } \right)^{1/\alpha} \right] \sum_{ i = 1}^{K} \exp\left[ -\left( \frac{ \theta_i }{ z_i } \right)^{ 1/\alpha} \right] \nonumber\\
		&  + \exp \left[ -\sum_{ i = K+1}^{ n }\left( \frac{ \theta_i }{ z_i } \right)^{1/\alpha} \right] \sum_{ 1 < i < j \le K } \left\{ \exp \left[ - \left( \frac{ \theta_i }{ z_i } \right)^{ 1/\alpha} - \left( \frac{ \theta_j }{ z_j } \right)^{ 1/\alpha } \right] \right \} \nonumber \\[0.5em]
		& + \cdots + (-1)^K \exp\left[ - \sum_{ i = 1 }^{ n }\left( \frac{ \theta_i }{ z_i } \right)^{ 1/\alpha} \right] p( \bA | \alpha) d\bA.
\end{align}

Consider the first term in the summation,

\begin{align}
	\int \exp \left\{ -\sum_{ i = K+1}^{ n }\left( \frac{ \theta_i }{ z_i } \right)^{1/\alpha} \right\} p( \bA | \alpha) d\bA &= \int \exp \left\{ - \sum_{ i = K + 1 }^n \left( \frac{ \left[ \sum_{ l = 1 }^L  A_l w_{l}(\bs_i)^{1/\alpha} \right)^\alpha }{ z_i} \right]^{ 1/\alpha } \right \} p( \bA | \alpha) d\bA \nonumber \\[0.5em]
	 &= \int \exp \left\{ -\sum_{ i = K + 1}^n \sum_{ l = 1}^L A_l \left( \frac{ w_l (\bs_i) }{ z_i } \right)^{1/\alpha} \right \} p( \bA | \alpha) d\bA \nonumber \\[0.5em]
	 &=\exp\left\{-\sum_{ l = 1}^L \left[ \sum_{ i = K + 1 }^n \left( \frac{ w_l(\bs_i)}{ z_i} \right)^{1/\alpha} \right]^\alpha \right\}.
\end{align}

The remaining terms in equation (\ref{joint}) are straightforward to obtain, and after integrating out the random effect, the joint density is the density given in (\ref{pmf}).

\subsection{Derivation of the $\chi$ statistic}\label{a:chi}
\begin{align} \label{chi}
  \chi &= \lim_{p \rightarrow 0 }\text{P}(Y_i = 1|Y_j = 1)\nonumber \\
   &= \lim_{p \rightarrow \infty }\frac{ p + p - \left( 1 - \exp \left \{ - \sum_{ l = 1 }^{ L } \left[ \left( -\log(1-p) w_{l}(\bs_i) \right)^{ 1/\alpha } + \left( -\log(1 - p) w_{ l}(\bs_j) \right)^{ 1/\alpha } \right]^{\alpha} \right \} \right) }{ p } \nonumber \\[0.5em]
  &= \lim_{p \rightarrow 0 } \frac{ 2p - \left( 1 - \exp \left \{ \log(1-p) \sum_{ l = 1 }^{ L } \left[  w_{l}(\bs_i) ^{ 1/\alpha } +  w_{ l}(\bs_j) ^{ 1/\alpha } \right]^{\alpha} \right \} \right) }{ p } \nonumber \\[0.5em]
  &= \lim_{p \rightarrow 0 } \frac{ 2p - \left( 1 - (1-p)^{ \sum_{ l = 1 }^{ L } \left[ \left( w_{l}(\bs_i) \right)^{ 1/\alpha } + \left( w_{ l}(\bs_j) \right)^{ 1/\alpha } \right]^\alpha } \right) }{ p } \nonumber \\[0.5em]
  &=\lim_{p \rightarrow 0 } 2 -\sum_{ l = 1 }^{ L } \left[ w_{l}(\bs_i) ^{ 1/\alpha } +  w_{ l} (\bs_j)^{ 1/\alpha } \right]^\alpha ( 1 - p)^{ -1 + \sum_{ l = 1 }^{ L } \left[  w_{l}(\bs_i) ^{ 1/\alpha } +  w_{ l}(\bs_j)^{ 1/\alpha } \right]^\alpha } \nonumber \\[0.5em]
  &= 2 -  \sum_{ l = 1 }^{ L } \left[ w_{l}(\bs_i)^{ 1/\alpha } +  w_{ l}(\bs_j) ^{ 1/\alpha } \right]^\alpha.
\end{align}

\begin{singlespace}
\bibliographystyle{rss}
\bibliography{library}
\end{singlespace}


\end{document}

