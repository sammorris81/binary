\documentclass[11pt]{article}
\usepackage{amssymb, amsthm, amsmath}
\usepackage{bm}
\usepackage{graphicx}
\usepackage[authoryear]{natbib}
\usepackage{bm}
\usepackage{verbatim}
\usepackage{lineno}
\usepackage{times}
\usepackage{soul}
\usepackage{color}
\usepackage{enumitem}
\usepackage{setspace}
\usepackage{times}
\usepackage{changepage}

\usepackage[left=1in,top=1in,right=1in]{geometry}
\pdfpageheight 11in
\pdfpagewidth 8.5in
\linespread{2.0}
\newcommand{\btheta}{ \mbox{\boldmath $\theta$}}
\newcommand{\bmu}{ \mbox{\boldmath $\mu$}}
\newcommand{\balpha}{ \mbox{\boldmath $\alpha$}}
\newcommand{\bbeta}{ \mbox{\boldmath $\beta$}}
\newcommand{\bdelta}{ \mbox{\boldmath $\delta$}}
\newcommand{\blambda}{ \mbox{\boldmath $\lambda$}}
\newcommand{\bgamma}{ \mbox{\boldmath $\gamma$}}
\newcommand{\brho}{ \mbox{\boldmath $\rho$}}
\newcommand{\bpsi}{ \mbox{\boldmath $\psi$}}
\newcommand{\bepsilon}{ \mbox{\boldmath $\epsilon$}}
\newcommand{\bomega}{ \mbox{\boldmath $\omega$}}
\newcommand{\bOmega}{ \mbox{\boldmath $\Omega$}}
\newcommand{\bDelta}{ \mbox{\boldmath $\Delta$}}
\newcommand{\bSigma}{ \mbox{\boldmath $\Sigma$}}
\newcommand{\bPsi}{\mbox{\boldmath $\Psi$}}
\newcommand{\bOne}{\mbox{\boldmath $1$}}
\newcommand{\omu}{\overline{\mu}}
\newcommand{\oSigma}{\overline{\Sigma}}
\newcommand{\Yt}{{\tilde Y}}
\newcommand{\bA}{ \mbox{\bf A}}
\newcommand{\bP}{ \mbox{\bf P}}
\newcommand{\bx}{ \mbox{\bf x}}
\newcommand{\bX}{ \mbox{\bf X}}
\newcommand{\bB}{ \mbox{\bf B}}
\newcommand{\bZ}{ \mbox{\bf Z}}
\newcommand{\by}{ \mbox{\bf y}}
\newcommand{\bY}{ \mbox{\bf Y}}
\newcommand{\bz}{ \mbox{\bf z}}
\newcommand{\bh}{ \mbox{\bf h}}
\renewcommand{\bm}{ \mbox{\bf m}}
\newcommand{\br}{ \mbox{\bf r}}
\newcommand{\bt}{ \mbox{\bf t}}
\newcommand{\bs}{ \mbox{\bf s}}
\newcommand{\bb}{ \mbox{\bf b}}
\newcommand{\bL}{ \mbox{\bf L}}
\newcommand{\bu}{ \mbox{\bf u}}
\newcommand{\bv}{ \mbox{\bf v}}
\newcommand{\bV}{ \mbox{\bf V}}
\newcommand{\bW}{ \mbox{\bf W}}
\newcommand{\bG}{ \mbox{\bf G}}
\newcommand{\bH}{ \mbox{\bf H}}
\newcommand{\bw}{ \mbox{\bf w}}
\newcommand{\bo}{ \mbox{\bf o}}
\newcommand{\bfe}{ \mbox{\bf e}}
\newcommand{\iid}{\stackrel{iid}{\sim}}
\newcommand{\ind}{\stackrel{ind}{\sim}}
\newcommand{\dd}{\; \text{d} }
\newcommand{\ddd}{\text{d} }
\newcommand{\indep}{\stackrel{indep}{\sim}}
\newcommand{\converged}{\stackrel{d}{\rightarrow}}
\newcommand{\calR}{{\cal R}}
\newcommand{\calG}{{\cal G}}
\newcommand{\calD}{{\cal D}}
\newcommand{\calS}{{\cal S}}
\newcommand{\calB}{{\cal B}}
\newcommand{\calA}{{\cal A}}
\newcommand{\calT}{{\cal T}}
\newcommand{\calO}{{\cal O}}
\newcommand{\argmax}{{\mathop{\rm arg\, max}}}
\newcommand{\argmin}{{\mathop{\rm arg\, min}}}
\newcommand{\Frechet}{\mbox{Fr$\acute{\mbox{e}}$chet }}
\newcommand{\Matern}{\mbox{Mat$\acute{\mbox{e}}$rn }}
\newcommand{\ballunion}{B_a(\bs_1) \cup B_b(\bs_2) }

\newcommand{\beq}{ \begin{equation}}
\newcommand{\eeq}{ \end{equation}}
\newcommand{\beqn}{ \begin{eqnarray}}
\newcommand{\eeqn}{ \end{eqnarray}}

\newcommand*\patchAmsMathEnvironmentForLineno[1]{%
  \expandafter\let\csname old#1\expandafter\endcsname\csname #1\endcsname
  \expandafter\let\csname oldend#1\expandafter\endcsname\csname end#1\endcsname
  \renewenvironment{#1}%
     {\linenomath\csname old#1\endcsname}%
     {\csname oldend#1\endcsname\endlinenomath}}%
\newcommand*\patchBothAmsMathEnvironmentsForLineno[1]{%
  \patchAmsMathEnvironmentForLineno{#1}%
  \patchAmsMathEnvironmentForLineno{#1*}}%
\AtBeginDocument{%
\patchBothAmsMathEnvironmentsForLineno{equation}%
\patchBothAmsMathEnvironmentsForLineno{align}%
\patchBothAmsMathEnvironmentsForLineno{flalign}%
\patchBothAmsMathEnvironmentsForLineno{alignat}%
\patchBothAmsMathEnvironmentsForLineno{gather}%
\patchBothAmsMathEnvironmentsForLineno{multline}%
}



\begin{document}\linenumbers

\begin{center}
{\Large {\bf A spatial model for rare binary events}}\\
\today
\end{center}

\section{Introduction}\label{s:intro}

The goal in binary regression is to relate a latent variable to a response using a link function.
% $g$ so that P$(Y_i=1) = \pi_i= g(\bX_i \bbeta)$, where $\bX_i$ is the vector of covariates for observation $i$, and $\bbeta$ is the $p$-vector of regression coefficients.
Two common examples of binary regression include logistic regression
% with $\pi_i = \frac{ \exp{\bX_i \bbeta} }{1 + \exp{\bX_i \bbeta}}$
and probit regression.
% with $\pi_i = \Phi(\bX_i \bbeta)$ where $\Phi(\cdot)$ represents the standard normal distribution function.
The link functions for logistic and probit regression are symmetric, so they may not be well-suited for asymmetric data.
% In the case that $\xi = 0$, this is the complementary log-log (cloglog) link function.
An asymmetric alternative to these link functions is the complementary log-log (cloglog) link function.
More recently, \citet{Wang2010} introduced the generalized extreme value (GEV) link function for rare binary data.
The GEV link function introduces a new shape parameter to the link function that controls the degree of asymmetry.
The cloglog link is a special case of the GEV link function when the shape parameter is 0.
% where

\hl{Want to make the case in this paragraph that spatial logistic and probit models are not appropriate because asymptotic dependence is 0.}
Spatial logistic and probit models are commonly presented using a hierarchical model \hl{citation}.
In the hierarchical framework, spatial dependence is typically modeled with an underlying latent Gaussian process, and conditioned on this process, observations are independent.
However, if the latent variable is assumed to follow a GEV marginally, then a Gaussian process may not be appropriate to describe the dependence due to the fact that they do not demonstrate asymptotic dependence regardless of the strength of the dependence in the bulk of the data.
As an alternative to the Gaussian process, we propose using a latent max-stable process because it allows for asymptotic dependence \hl{citation}.
Max-stable processes are extremely flexible, but are often challenging to work with because very few finite dimensional representations exist in more than two or three dimensions.


\hl{Paragraph outlining the structure of the paper}

\section{Binary regression using the GEV link}\label{s:rarebinary}

Here, we provide a brief review of the the GEV link of \citet{Wang2010}.
Let $Y_i \in \{0, 1\}, i = 1, \ldots, n$ be a collection of i.i.d. binary responses.
We assume that $Y_i = I(z_i > 0)$ where $I(\cdot)$ is an indicator function, $z_i = [1 - \xi \bX_i \bbeta]^{1 / \xi}$ is a latent variable following a GEV$(1, 1, 1)$ distribution, $\bX_i$ be the associated $p$-vector of covariates with first element equal to one for the intercept, and $\bbeta$ is a $p$-vector of regression coefficients.
Therefore, the marginal probability of an event is given by
\begin{align} \label{eq:gevlink}
  \pi_i= 1 - \exp \left( -\frac{ 1 }{ z_i } \right).
\end{align}
Although this link was selected by \citeauthor{Wang2010} based on its ability to handle asymmetry, the GEV distribution is one of the primary distributions used for modeling extremes.
Therefore, it should also be suitable to model phenomena that are rare.

\section{Spatial dependence for binary regression}

In many binary regression applications, spatial dependence is handled using a hierarchical model.
Logistic and probit regression typically incorporate a hierarchical model that is based on a Gaussian process.
We present this in section \ref{s:nonrarespatial}.

\subsection{Non-rare data}\label{s:nonrarespatial}

In this section we present the spatial logistic and probit models.

In the case that $n$ is large, low-rank predictive process models can be used to ease the computation.

\subsection{Rare binary data}\label{s:rarebinary}

In this section we extend the GEV link function to allow for spatial dependence.
Let $Y(\bs)$ be the observation at spatial location $\bs$ in a spatial domain of interest $\calD \in \calR^2$.
As before, we assume $Y(\bs) = I[z(\bs) > 0]$ where $z(\bs) = [1 - \xi \bX(\bs) \bbeta]^{1 / \xi}$.
Although $\bbeta$ and $\xi$ could be permitted to vary across space, we assume that they are constant across $\calD$.
As with logistic and probit regression, we also account for the spatial dependence using a hierarchical framework by assuming that the $Z(\bs)$ come from a realization of a max-stable process.

To incorporate spatial dependence into the model, we consider the hierarchical max-stable process of \citet{Reich2012}.
For a finite collection of locations $\bs_1, \ldots, \bs_n,$ denote the vector of observations $\bY = [Y(\bs_1), \ldots, Y(\bs_n)]^T$.
The spatial dependence is determined by the joint distribution of $\bz = [z(\bs_1),\ldots, z(\bs_n)]^T$,
\begin{align}\label{jointCDF}
  G(\bz) = \mbox{P}[Z(\bs_1) < z(\bs_1), \ldots, Z(\bs_n) < z(\bs_n)] = \exp\left\{-\sum_{l=1}^L\left[\sum_{i=1}^n\left(\frac{w_{l}(\bs_i)}{z(\bs_i)}\right)^{1/\alpha}\right]^{\alpha}\right\},
\end{align}
where $w_{l}(\bs_i)$ are a set of weights that determine the spatial dependence structure and are discussed further in Section \ref{s:weights}, and $\alpha\in(0,1)$ determines the strength of dependence, with $\alpha$ near zero giving strong dependence and $\alpha=1$ giving joint independence.
This is a special case of the multivariate GEV distribution with asymmetric Laplace dependence function \citep{Tawn1990}.
One nice feature to this representation for the max-stable process is that the lower-dimensional marginal distributions also follow a multivariate extreme value distribution.
More importantly, at a single site $i$, the marginal distribution gives \mbox{$P[Y(\bs_i) = 1] = 1 - \exp\left[ -\frac{ 1 }{ z(\bs_i)} \right]$} which is the same as the marginal distributions given by \citet{Wang2010}.

\subsection{Weight functions}\label{s:weights}
The weights $w_{l}(\bs_i)$ in (\ref{jointCDF}) should vary smoothly across space to induce spatial dependence.
Many weight functions are possible, but the weights must be constrained so that $\sum_{l=1}^L w_{l}(\bs_i)=1$ for all $i=1,\ldots,n$ to preserve the marginal GEV distribution.
For example, \cite{Reich2012} take the weights to be scaled Gaussian kernels with knots $\bv_l$, that is
\begin{align}\label{w}
   w_{l}(\bs_i) = \frac{\exp\left[-0.5\left(||\bs_i-\bv_l||/\rho\right)^2\right]}
                 {\sum_{j=1}^L\exp\left[-0.5\left(||\bs_i-\bv_j||/\rho\right)^2\right]}.
\end{align}
The kernel bandwidth $\rho>0$ determines the spatial range of the dependence, with large $\rho$ giving long-range dependence and vice versa.

\section{Joint distribution}\label{s:multivariate}
The joint likelihood of $Y$ is computationally challenging to obtain.
In this paper, we combine the random effect representation with MCMC; however, when exact expressions of the likelihood are available, better methods could be used.
In section \ref{s:bivariate}, we give an exact expression in the case where there are only two spatial locations which is useful for constructing a pairwise composite likelihood and studying spatial dependence.
For more than two locations, we are also able to compute the exact likelihood when the number of locations is large but the number of events is small, as might be expected for very rare events (see Appendix \ref{a:likelihoodderivation}).

\subsection{Bivariate distribution}\label{s:bivariate}
Then in a bivariate setting, the probability mass function as a function of $\alpha$ is
\begin{adjustwidth}{-0.25in}{}
{\footnotesize
\begin{align} \label{biv}
  \text{P}[Y(\bs_i), Y(\bs_j)] = \left\{ \begin{array}{ll}
    \exp \left\{ - \sum_{ l = 1 }^{ L } \left[ \left( \frac{ w_{ l }(\bs_i) }{ z(\bs_i) } \right)^{1/\alpha} + \left( \frac{ w_{l }(\bs_j)}{z(\bs_j)} \right)^{1/\alpha} \right]^{\alpha} \right\} &Y(\bs_i) = 0, Y(\bs_j) = 0\\
    \frac{ 1 }{z(\bs_i)} - \exp \left\{ - \sum_{ l = 1 }^{ L } \left[ \left( \frac{ w_{ l }(\bs_i) }{ z(\bs_i) } \right)^{1/\alpha} + \left( \frac{ w_{l }(\bs_j)}{z(\bs_j)} \right)^{1/\alpha} \right]^{\alpha} \right\} &Y(\bs_i) = 1, Y(\bs_j) = 0 \\
    1 - \exp\left\{ - \frac{ 1 }{ z(\bs_i) } \right\} - \exp \left\{ - \frac{ 1 }{z(\bs_j)} \right\} + \exp \left\{ - \sum_{ l = 1 }^{ L } \left[ \left( \frac{ w_{ l }(\bs_i) }{ z(\bs_i) } \right)^{1/\alpha} + \left( \frac{ w_{l }(\bs_j)}{z(\bs_j)} \right)^{1/\alpha} \right]^{\alpha} \right\} &Y(\bs_i) = 1, Y(\bs_j) = 1
  \end{array} \right.
\end{align}
}
\end{adjustwidth}

\section{Quantifying spatial dependence}
\hl{I still need to incorporate Brian's suggestions here}
In the literature on extremes, one common metric to describe the bivariate dependence is the $\chi$ statistic of \citet{Coles1999}.
The $\chi$ statistic between two observations $z_1$ and $z_2$ is given by
\begin{align}
  \chi(\bs_1, \bs_2) = \lim_{c \rightarrow \infty} = P(Z_1 > c | Z_2 > c).
\end{align}
However, in this latent variable approach, $\lim_{c \rightarrow \infty}$ may not be the most reasonable metric because the observed data are a series of zeros and ones.
Therefore, we chose the $\kappa$ statistic of \citet{Cohen1960} defined by
\begin{align}
  \kappa = \frac{P(A) - P(E)}{1 - P(E)}
\end{align}
where $P(A)$ is the joint probability of agreement and $P(E)$ is the joint probability of agreement under an assumption of independence.
We believe this measure of dependence to be reasonable because,
\begin{align}
  \lim_{\beta_0 \rightarrow \infty} \kappa(h) = \chi(h) = 2 - \vartheta(\bs_i, \bs_j)
\end{align}
where $\beta_0$ is the intercept from $\bX^T\bbeta$ and $\vartheta (\bs_i, \bs_j) = \sum_{ l = 1 }^{ L } \left[ w_{l}(\bs_i)^{ 1/\alpha } +  w_{ l}(\bs_j)^{ 1/\alpha } \right]^\alpha$ is the pairwise extremal coefficient given by \citet{Reich2012} (see Appendix \ref{a:chi}).
In the case of complete dependence, $\kappa = 1$, and in the case of complete independence, $\kappa = 0$.

\section{Computation}\label{s:comp}
For small $K$ we can evaluate the likelihood directly.
In the random effects model, the expression for the joint density conditional on $\theta$ is
\begin{align}
	P(Y_1=y_1,\ldots,Y_n=y_n) = \prod_{ i = 1 }^{ n } \left[ \exp \left\{ -\sum_{ l = 1 }^{L} A_l \left( \frac{ w_{l}(\bs_i) }{ z_i } \right)^{ 1/\alpha} \right\} \right]^{ 1 - Y_i } \left[ 1 - \exp \left\{ -\sum_{ l = 1 }^{L} A_l \left( \frac{ w_{l}(\bs_i) }{ z_i } \right)^{ 1/\alpha} \right\} \right]^{ Y_i }.
\end{align}

\section{Simulation study}\label{s:sim}
For our simulation study, we generate $n_m = 100$ datasets under 4 different settings to explore the impact of rareness of observations, sample size, and knot spacing.
We consider two degrees of rareness $\pi = 0.01, 0.05$ and two sample sizes $n_s = 1000, 2000$.
For the different knot spacings, we use knots in $[0, 1] \times [0, 1]$ on a $21 \times 21$ grid and $31 \times 31$ grid.
For each dataset, we fit the model using three different methods, spatial logistic regression, spatial probit regression, and the proposed spatial GEV method.
In each case, we fit the model using Bayesian methods with proper, but fairly uninformative priors.
For each method, we fit the model using 75\% of the observations as a training set, and the remaining observations are used as a validation set to assess the model's predictive power.

\section{Data analysis}\label{s:analysis}
For the data analysis, we consider data from the eBirds dataset, a citizen-based observation network of bird sitings in the United States \citep{Sullivan2009}.
The data are publicly available from {\tt http://ebird.org}.
We use data from 2002, and focus specifically on cattle egrets and vesper sparrows.

\section{Conclusions}\label{s:con}

\section*{Acknowledgments}

\appendix
\section{Appendices}

\subsection{Derivation of the likelihood} \label{a:likelihoodderivation}
We use the hierarchical max-stable spatial model given by \citet{Reich2012}. If at each margin, $Z_i \sim $ GEV$(1,1,1)$, then $Z_i | \theta_i \indep $ GEV$(\theta, \alpha \theta, \alpha)$. We reorder the data such that $Y_1=\ldots=Y_K=1$, and $Y_{K+1} = \ldots = Y_n = 0$. Then the joint likelihood conditional on the random effect $\theta$ is

\begin{align} \label{joint_cond}
	P(Y_1=y_1,\ldots,Y_n=y_n) =& \prod_{ i \le K } \left\{ 1 - \exp \left[ - \left( \frac{ \theta_i }{ z_i } \right)^{ 1/\alpha} \right] \right \} \prod_{ i > K } \exp \left[ -\left( \frac{ \theta_i }{ z_i } \right)^{1/\alpha} \right] \nonumber \\[0.5em]
		=& \exp \left[ -\sum_{ i = K+1}^{ n }\left( \frac{ \theta_i }{ z_i } \right)^{1/\alpha} \right] - \exp \left[ -\sum_{ i = K+1}^{ n }\left( \frac{ \theta_i }{ z_i } \right)^{1/\alpha} \right] \sum_{ i = 1}^{K} \exp\left[ -\left( \frac{ \theta_i }{ z_i } \right)^{ 1/\alpha} \right] \nonumber\\
		&  + \exp \left[ -\sum_{ i = K+1}^{ n }\left( \frac{ \theta_i }{ z_i } \right)^{1/\alpha} \right] \sum_{ 1 < i < j \le K } \left\{ \exp \left[ - \left( \frac{ \theta_i }{ z_i } \right)^{ 1/\alpha} - \left( \frac{ \theta_j }{ z_j } \right)^{ 1/\alpha } \right] \right \} \nonumber \\[0.5em]
		& + \cdots + (-1)^K \exp\left[ - \sum_{ i = 1 }^{ n }\left( \frac{ \theta_i }{ z_i } \right)^{ 1/\alpha} \right]
\end{align}

Finally marginalizing over the random effect, we obtain

\begin{align} \label{joint}
    P(Y_1=y_1,\ldots,Y_n=y_n) =&\int G(\bz | \bA) p( \bA | \alpha) d\bA. \nonumber\\[0.5em]
			=& \int \exp \left[ -\sum_{ i = K+1}^{ n }\left( \frac{ \theta_i }{ z_i } \right)^{1/\alpha} \right] - \exp \left[ -\sum_{ i = K+1}^{ n }\left( \frac{ \theta_i }{ z_i } \right)^{1/\alpha} \right] \sum_{ i = 1}^{K} \exp\left[ -\left( \frac{ \theta_i }{ z_i } \right)^{ 1/\alpha} \right] \nonumber\\
		&  + \exp \left[ -\sum_{ i = K+1}^{ n }\left( \frac{ \theta_i }{ z_i } \right)^{1/\alpha} \right] \sum_{ 1 < i < j \le K } \left\{ \exp \left[ - \left( \frac{ \theta_i }{ z_i } \right)^{ 1/\alpha} - \left( \frac{ \theta_j }{ z_j } \right)^{ 1/\alpha } \right] \right \} \nonumber \\[0.5em]
		& + \cdots + (-1)^K \exp\left[ - \sum_{ i = 1 }^{ n }\left( \frac{ \theta_i }{ z_i } \right)^{ 1/\alpha} \right] p( \bA | \alpha) d\bA.
\end{align}

Consider the first term in the summation,

\begin{align}
	\int \exp \left\{ -\sum_{ i = K+1}^{ n }\left( \frac{ \theta_i }{ z_i } \right)^{1/\alpha} \right\} p( \bA | \alpha) d\bA &= \int \exp \left\{ - \sum_{ i = K + 1 }^n \left( \frac{ \left[ \sum_{ l = 1 }^L  A_l w_{l}(\bs_i)^{1/\alpha} \right)^\alpha }{ z_i} \right]^{ 1/\alpha } \right \} p( \bA | \alpha) d\bA \nonumber \\[0.5em]
	 &= \int \exp \left\{ -\sum_{ i = K + 1}^n \sum_{ l = 1}^L A_l \left( \frac{ w_l (\bs_i) }{ z_i } \right)^{1/\alpha} \right \} p( \bA | \alpha) d\bA \nonumber \\[0.5em]
	 &=\exp\left\{-\sum_{ l = 1}^L \left[ \sum_{ i = K + 1 }^n \left( \frac{ w_l(\bs_i)}{ z_i} \right)^{1/\alpha} \right]^\alpha \right\}.
\end{align}

The remaining terms in equation (\ref{joint}) are straightforward to obtain, and after integrating out the random effect, the joint density for $K = 0, 1, 2$ is given by
\begin{align}\label{pmf}
  P(Y_1=y_1,\ldots,Y_n=y_n) =  \left\{
    \begin{array}{ll}
      G(\bz) & K=0 \\
      G(\bz_{(1)})-G(\bz) & K=1 \\
      G(\bz_{(12)})-G(\bz_{(1)})-G(\bz_{(2)})+G(\bz) & K=2
    \end{array}
  \right.
\end{align}
where
\begin{align*}
  G[\bz_{(1)}] &= P[Z(\bs_2)<z(\bs_2),\ldots,Z(\bs_n)<z(\bs_n)] \\
  G[\bz_{(2)}] &= P[Z(\bs_1)<z(\bs_1),Z(\bs_3)<z(\bs_3),\ldots,Z(\bs_n)<z(\bs_n)]\\
  G[\bz_{(12)}] &= P[Z(\bs_3)<z(\bs_3),\ldots,Z(\bs_n)<z(\bs_n)].
\end{align*}
Similar expressions can be derived for all $K$, but become cumbersome for large $K$.

\subsection{Derivation of the $\chi$ statistic}\label{a:chi}
\begin{align} \label{chi}
  \chi &= \lim_{p \rightarrow 0 }\text{P}(Y_i = 1|Y_j = 1)\nonumber \\
   &= \lim_{p \rightarrow \infty }\frac{ p + p - \left( 1 - \exp \left \{ - \sum_{ l = 1 }^{ L } \left[ \left( -\log(1-p) w_{l}(\bs_i) \right)^{ 1/\alpha } + \left( -\log(1 - p) w_{ l}(\bs_j) \right)^{ 1/\alpha } \right]^{\alpha} \right \} \right) }{ p } \nonumber \\[0.5em]
  &= \lim_{p \rightarrow 0 } \frac{ 2p - \left( 1 - \exp \left \{ \log(1-p) \sum_{ l = 1 }^{ L } \left[  w_{l}(\bs_i) ^{ 1/\alpha } +  w_{ l}(\bs_j) ^{ 1/\alpha } \right]^{\alpha} \right \} \right) }{ p } \nonumber \\[0.5em]
  &= \lim_{p \rightarrow 0 } \frac{ 2p - \left( 1 - (1-p)^{ \sum_{ l = 1 }^{ L } \left[ \left( w_{l}(\bs_i) \right)^{ 1/\alpha } + \left( w_{ l}(\bs_j) \right)^{ 1/\alpha } \right]^\alpha } \right) }{ p } \nonumber \\[0.5em]
  &=\lim_{p \rightarrow 0 } 2 -\sum_{ l = 1 }^{ L } \left[ w_{l}(\bs_i) ^{ 1/\alpha } +  w_{ l} (\bs_j)^{ 1/\alpha } \right]^\alpha ( 1 - p)^{ -1 + \sum_{ l = 1 }^{ L } \left[  w_{l}(\bs_i) ^{ 1/\alpha } +  w_{ l}(\bs_j)^{ 1/\alpha } \right]^\alpha } \nonumber \\[0.5em]
  &= 2 -  \sum_{ l = 1 }^{ L } \left[ w_{l}(\bs_i)^{ 1/\alpha } +  w_{ l}(\bs_j) ^{ 1/\alpha } \right]^\alpha.
\end{align}

\begin{singlespace}
\bibliographystyle{rss}
\bibliography{library}
\end{singlespace}


\end{document}

